\documentclass[a4paper,12pt]{article}
\usepackage{graphicx,epsfig}
\usepackage[english]{babel}
\usepackage{latexsym}

\newcommand{\slashed}[1]{\hbox{{$#1$}\llap{$/$}}}

\begin{document}

\section{LV operator}
     The operator 
\[
  \frac{n^\mu} {4 M} 
    \bigl . \overline{\sigma}_\mu^{\dot{\alpha}\alpha}
   \int d^4 \theta \bar{S} e^V 
                  \bar{D}_{\dot{\alpha}} e^{-V} 
		  D_\alpha e^V S ~  
\]
    when expressed in components, looks as
\begin{eqnarray*}
% the 1st line
{\mathcal L} & =  \frac{n^\mu}{M} \Bigl [&
  i \bar{F} \mathcal{D}_\mu F +
  \frac{1}{2} i \bar{z} D \mathcal{D}_\mu z - 
  \frac{1}{2} i \mathcal{D}_\mu(\bar{z}) D z + 
  i \frac{\sqrt{2}}{4} \left[
             \overline{\psi\sigma^\mu}\lambda F -
             \overline{F\lambda\sigma^\mu} \psi
                       \right] + 
  \frac{1}{4}\overline{\psi\sigma^\mu}D\psi + \\
% the 2nd line
&&  + \frac{1}{4} \bar{z} \left\{ 
               \lambda\sigma^\mu\bar{\lambda} -
               \overline{\lambda\sigma^\mu}\lambda 
                       \right\} z - \\
% the 3rd line
&& - \frac{\sqrt{2}}{4} \left\{ 
                      \overline{\psi\sigma^\nu}\sigma^\mu 
                     \bar{\lambda}\mathcal{D}_\nu z +
                     \mathcal{D}_\nu(\bar{z})\lambda\sigma^\mu
                     \bar{\sigma}^\nu \psi
                     \right \}
 - \frac{\sqrt{2}}{2} \left\{ 
                     \mathcal{D}_\mu(\overline{\psi)\lambda} z + 
                     \bar{z} \lambda \mathcal{D}_\mu \psi 
                     \right\} + \\
% the 4rd line
&& + \frac{1}{2}\bar{\psi}\mathcal{D}_{(\mu}\mathcal{D}_{\nu)}
               \bar{\sigma}^\nu \psi -
  \frac{1}{8} \bar{\psi}\epsilon^{\mu\nu\rho\sigma}
              F_{\rho\sigma} \bar{\sigma}_\nu \psi +
  i \bar{z} \mathcal{D}^\nu \mathcal{D}_\mu \mathcal{D}_\nu z +
  \frac{1}{4} i \mathcal{D}_\nu (\bar{z}) \epsilon^{\mu\nu\rho\sigma}
              F_{\rho\sigma} z~~~
                               \Bigr ] 
\end{eqnarray*}

%%%
%%% Introduction of the operator P
%%%
\section{Operator $ \mathcal{P}_{\alpha\dot{\alpha}} $}
  We introduce the following differential operator 
$ \mathcal{P}_{\alpha\dot{\alpha}} $: 
\begin{eqnarray*}
%% first line
   \mathcal{P}_{\alpha\dot{\alpha}} & = & \left\{ \bar{D}_{\dot{\alpha}} 
D_\alpha \right \} + 
   \left\{ \bar{D}_{\dot{\alpha}} ( D_\alpha V ) \right \} + 
   \frac{1}{2} \left\{ \bar{D}_{\dot{\alpha}}  
             [ (D_\alpha V), V ] \right\} = \\
%% second line
 &  = & - 2 i \slashed{\mathcal{D}}_{\alpha\dot{\alpha}} 
            - 2 i \theta_\alpha \bar{\lambda}_{\dot{\alpha}} 
            - 2 i \bar{\theta}_{\dot{\alpha}} \lambda_\alpha - \\
%% third line
 &&       - 2 \theta_\alpha \bar{\theta}_{\dot{\alpha}} D 
      + i ( \slashed{\partial} (\bar{\slashed{v}}) \theta )_\alpha \bar{\theta}_{\dot{\alpha}}  
      + i \theta^\beta \bar{\theta}^{\dot{\beta}} 
            \slashed{\partial}_{\beta\dot{\alpha}} 
	     ( \slashed{v}_{\alpha\dot{\beta}} ) 
      + \frac{1}{2} \bar{\theta}_{\dot{\alpha}} 
          ( \sigma^\mu \bar{\sigma}^\nu \theta )_\alpha 
	  [ v_\mu v_\nu ] + \\
%% fourth line
  &&  + \theta^2 \slashed{\partial}_{\alpha\dot{\alpha}} ( \overline{\theta\lambda} )  
  + \theta^2 \bar{\theta}_{\dot{\alpha}} \slashed{\partial}_{\alpha\dot{\beta}} 
                 ( \bar{\lambda}^{\dot{\beta}} )
  + i \bar{\theta}_{\dot{\alpha}} \theta^2 
                 [ \slashed{v}_{\alpha\dot{\beta}} \bar{\lambda}^{\dot{\beta}} ]
  - \bar{\theta}^2 \theta^\beta \slashed{\partial}_{\beta\dot{\alpha}} ( \lambda_\alpha ) - \\
%% fifth line
  && - \frac{1}{2} i \theta^2 \bar{\theta}^2 \partial_{\alpha\dot{\alpha}} ( D ) 
     - \frac{1}{4} \theta^2 \bar{\theta}^2 ( \slashed{\partial} \bar{\slashed{v}}
                          \overleftarrow{\slashed{\partial}} )_{\alpha\dot{\alpha}} 
     - \frac{1}{8} i \theta^2 \bar{\theta}^2 
                  ( \sigma^\mu \bar{\sigma}^\nu \slashed{\partial} )_{\alpha\dot{\alpha}}
		      ( [v_\mu v_\nu] ) 
\end{eqnarray*}
  where we have the covariant derivative:
\[
   \slashed{\mathcal{D}}_{\alpha\dot{\alpha}} \equiv 
      \sigma^\mu_{\alpha\dot{\alpha}} 
      \left [  \partial_\mu + \frac{i}{2} v_\mu \right ]
\]

\section{Basic $|$-properties}
   Components of 
$ L $
   and
$ \bar{L} $:
\begin{eqnarray*}
    & L | = z                &  \bar{L} | = \bar{z} \\
    & D_\alpha L | = \sqrt{2} \psi_\alpha &
        D_{\dot{\alpha}} \bar{L} | = \sqrt{2} \bar{\psi}_{\dot{\alpha}} \\
    & D^2 L | = - 4 F        &  \overline{D^2 L} | = - 4 \bar{F} \\        
\end{eqnarray*}

%%%                                     V
%%% Now we introduce the components of e
%%%
   Components of 
   $ e^V $:
%\[
% e^V = 1 - \theta\sigma^\mu \bar{\theta} v_\mu 
%          + i \theta^2 \overline{\theta\lambda} 
%          - i \bar{\theta}^2 \theta\lambda 
%          + \frac{1}{2} \theta^2 \bar{\theta}^2 
%            ( D - \frac{1}{2} v_\mu v^\mu ) 
%\]
\begin{eqnarray*}
 &&  ~~~~e^V | = 1~~~~ D_\alpha e^V | = \bar{D}_{\dot{\alpha}} e^V | = 0  \\
 &&  D^2 e^V | = \bar{D}^2 e^V | = 0 ~~~~  
 D_\alpha \bar{D}_{\dot{\alpha}} | = - \slashed{v}_{\alpha\dot{\alpha}} \\
 &&  D^2 \bar{D}^{\dot{\alpha}} e^V | = - 4 i \bar{\lambda}^{\dot{\alpha}}~~~~~
     D^\alpha \bar{D}^2 e^V | = 4 i \lambda^\alpha \\
 && D^2 \bar{D}^2 e^V | = 8 \left( D - \frac{1}{2} v_\mu v^\mu \right) - 8 i \partial_\mu (v^\mu) 
\end{eqnarray*} 

%%% 
%%% Now go the components of P
%%%
  Components of   
$ \mathcal{P}_{\alpha\dot{\alpha}} $:
\begin{eqnarray*}
%% first line
  \mathcal{P}_{\alpha\dot{\alpha}} | = - 2 i \slashed{\mathcal{D}}_{\alpha\dot{\alpha}} &
  D_\beta \mathcal{P}_{\alpha\dot{\alpha}} | =
            2 i \epsilon_{\beta\alpha}\bar{\lambda}_{\dot{\alpha}} &
  \bar{D}^{\dot{\beta}} \mathcal{P}_{\alpha\dot{\alpha}} | =
       - 2 i \delta^{\dot{\beta}}_{~\dot{\alpha}} \lambda_\alpha \\
%% second line
  & D^2 \mathcal{P}_{\alpha\dot{\alpha}} | = 0 ~~~
  \bar{D}^2 \mathcal{P}_{\alpha\dot{\alpha}} \equiv 0 &
\end{eqnarray*}

  More properties of 
$ \mathcal{P}_{\alpha\dot{\alpha}} $:
\begin{eqnarray*}
%% first line
    D_\gamma \bar{D_{\dot{\varepsilon}}} \mathcal{P}_{\alpha\dot{\alpha}} | & = &
 2 \epsilon_{\alpha\gamma} \epsilon_{\dot{\varepsilon}\dot{\alpha}} D 
  + i \slashed{\partial}_{\alpha\dot{\beta}} ( \bar{\slashed{v}}^{\dot{\beta}}_{~\gamma} )
                \epsilon_{\dot{\varepsilon}\dot{\alpha}} 
  + i \slashed{\partial}_{\gamma\dot{\alpha}} ( \slashed{v}_{\alpha\dot{\varepsilon}} ) \\
%% second line
  && 
  -~ 2 \slashed{\partial}_{\gamma\dot{{\varepsilon}}}
          (\slashed{\mathcal{D}}_{\alpha\dot{\alpha}} )
  - \frac{1}{2} \epsilon_{\dot{\varepsilon}\dot{\alpha}}
                \left( \sigma^\mu \bar{\sigma}^\nu \right)_{\alpha\gamma}
                [v_\mu v_\nu] \\
%% third line
   D^2 \bar{D}^{\dot{\varepsilon}} \mathcal{P}_{\alpha\dot{\alpha}} | & = &
   4 \slashed{\partial}_\alpha^{~\dot{\varepsilon}} 
            ( \bar{\lambda}_{\dot{\alpha}} ) 
 - 4 \slashed{\partial}_{\alpha\dot{\alpha}} 
            ( \bar{\lambda}^{\dot{\varepsilon}} ) 
 - 4 \delta^{\dot{\varepsilon}}_{~\dot{\alpha}}
            \slashed{\partial}_{\alpha\dot{\beta}}
            (\bar{\lambda}^{\dot{\beta}} ) 
 - 4 i \delta^{\dot{\varepsilon}}_{~\dot{\alpha}} 
            [ \slashed{v}_{\alpha\dot{\beta}} \bar{\lambda}^{\dot{\beta}} ]
\end{eqnarray*}

%%%
%%% Now some lower-components of derivatives
%%%
  The following relations are useful when deriving components of a particular
expression:
\begin{eqnarray*}
%% first line
  D_\alpha \bar{D}_{\dot{\alpha}} | = &
       - \partial_\alpha \bar{\partial}_{\dot{\alpha}} 
       - i \slashed{\partial}_{\alpha{\dot{\alpha}}} 
  ~~~~~~~
  \bar{D}_{\dot{\alpha}}D_\alpha | = &
       - \bar{\partial}_{\dot{\alpha}} \partial_\alpha
       - i \slashed{\partial}_{\alpha{\dot{\alpha}}} \\
%% second line
  D^2 \bar{D}_{\dot{\beta}} | = &
       - \partial^2 \bar{\partial}_{\dot{\beta}} 
       - 2 i \partial^\alpha \slashed{\partial}_{\alpha\dot{\beta}}
  ~~~
  D^\alpha \bar{D}^2 | =  &
       \partial^\alpha \bar{\partial}^2  
       - 2 i \overline{\partial_{\dot{\beta}}
                       \slashed{\partial}^{\dot{\beta}\alpha}} \\
%% third line
  & D^2 \bar{D}^2 | =
          \partial^2 \bar{\partial}^2  
    + 4 i \partial \slashed{\partial} \bar{\partial} 
    + 4 \Box &
\end{eqnarray*}
\end{document}
